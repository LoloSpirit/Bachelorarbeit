\chapter*{Zusammenfassung}
Der Trend zu großen Satellitenkonstellationen im niedrigen Erdorbit und deren Kommerzialisierung verändert die wirtschaftlichen Anforderungen an elektrische Antriebssysteme. Während Xenon für Ionentriebwerke als Standardtreibstoff gilt, machen seine hohen Kosten und begrenzte Verfügbarkeit Alternativen zunehmend attraktiver. Deswegen rückt die Erforschung kostengünstigerer und nachhaltigerer Treibstoffe in den Fokus. Um mögliche Alternativen zu untersuchen, wird in dieser Arbeit ein Stoßionisations-Flugzeit-Massenspektrometer mit Extraktionsverzögerung, genannt \textsc{Zero-B}, optimiert. 

Dazu wird ein neuer Mikrokanalplatten-Detektor mit größerem Durchmesser und höherer Ortsauflösung in die bestehende Zero-B-Anlage integriert und validiert. Zur Kalibration erfolgt die Aufnahme von Flugzeitspektren von Argon bei unterschiedlichen Elektronenenergien und ihre Transformation zu Masse-zu-Ladungsspektren. Die ermittelten relativen Häufigkeiten werden mit Werten von Straub et al. \cite{Straub} verglichen und zeigen eine gute Übereinstimmung. Zusätzlich erfolgt eine Überprüfung der Positionsdaten, um die Entstehungsorte der Ionen abzubilden und das Elektronenstrahlprofil zu ermitteln. Die Auswertung ergibt einen starken Zusammenhang der Entstehungsorte mit dem Strahl. Durch die Variation der Extraktionsverzögerung wird ein optimaler Wert von 600 ns bestimmt, der die Auflösung der Anlage bei kleinen Elektronenenergien deutlich verbessert.

Für eine weitere Optimierung der Anlage wird eine Simulation in \textsc{Simion} implementiert, die die Ionenoptik der Anlage abbildet. Die experimentellen Daten werden anhand der Simulation überprüft und zeigen, dass das physikalische Verständnis ausreichend ist, um diese zu replizieren. Eine experimentell ermittelte Aufweitung des Strahlabbilds auf dem Detektor sowie dessen Überhöhung an den Rändern werden zudem dokumentiert und mithilfe der Simulation untersucht, können jedoch nicht vollständig erklärt werden. Bei Untersuchung und Simulation der Ionenoptik der Anlage wird festgestellt, dass eine kleinere Detektordistanz von 210 mm und die damit einhergehende kürzere Flugzeit sich sehr positiv auf das Auflösungsvermögen der Anlage auswirken würde. Dieses wird quantifiziert und der Zusammenhang zwischen Auflösung und Masse der Ionen dargestellt und mit dem Experiment verglichen. Die Simulation erweist sich als nützliches Werkzeug zur Optimierung der Anlage und einige Vorschläge zur weiteren Verbesserung werden diskutiert.

\chapter*{Abstract}
With the rise and commercialization of satellites operating in low earth orbit, the demand for cost-effective and sustainable alternatives to xenon as a propellant for gridded ion thrusters is increasing. To investigate alternative propellants, an electron-impact ionization time-of-flight mass spectrometer with extraction delay, called \textsc{Zero-B}, is optimized in this thesis. A larger microchannel plate detector with higher spatial resolution is installed and validated.

For calibration, time-of-flight spectra of argon are recorded at different electron energies and transformed into mass-to-charge spectra. The relative frequencies of ions are compared with values from Straub et al. \cite{Straub} and show good agreement. Additionally, the position data is analyzed to map the origin of ions and determine the electron beam profile. The evaluation reveals a strong correlation between the ion origin and the beam. By varying the extraction delay, an optimal value of 600 ns is determined, which significantly enhances the system’s resolution at low electron energies.

For further improvement, a simulation in \textsc{SIMION} is implemented to model the ion optics of the system. The experimental data are compared with the simulation and confirm that the physical understanding of the system is sufficient to replicate it. An experimentally determined broadening of the beam image on the detector and an increased intensity at its edges are documented and investigated using the simulation. However, these effects cannot yet be fully explained. By examining and simulating the ion optics of the system, it is shown that a smaller detector distance of 210 mm and the resulting shorter flight time would significantly improve the system’s resolution. The resolution is quantified, and the relationship between resolution and ion mass is analyzed and compared with the experiment. The simulation proves to be a useful tool for optimizing the system. Suggestions for further improvement of experiments are discussed.