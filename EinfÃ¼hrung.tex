\chapter{Einführung}
In unserer vernetzten Welt haben erdnahe Satelliten eine immer größere Bedeutung in der globalen Infrastruktur. Sie finden vor allem Anwendung in der Kommunikation und Erdbeobachtung. Satelliten im nahen Erdorbit (\textit{low earth orbit}, LEO) haben den Vorteil, dass sie deutlich geringere Signallaufzeiten gegenüber geostationären Satelliten aufweisen. Das ermöglicht schnellere Kommunikation. Auch die Aufnahme hochauflösender Messungen der Erdoberfläche wird durch die Flughöhe begünstigt und ist in einer höheren Wiederholrate aufgrund der kürzeren Umlaufdauer möglich. Für den Einsatz im LEO werden immer mehr kleine, kostengünstige Satelliten produziert. Vor allem durch die Kommerzialisierung der Raumfahrtbranche, die im letzten Jahrzehnt durch Unternehmen wie SpaceX und Blue Origin vorangetrieben wurde, sind die Kosten für den Start von Satelliten gesunken und die Zahl der Starts enorm gestiegen. Jeder dieser Satelliten benötigt eine Lageregelung und damit ein Antriebssystem, welches kostengünstig im LEO über lange Zeiten dem atmosphärischen Widerstand entgegenwirken kann.

Damit sind elektrische Antriebssysteme in den Vordergrund gerückt, da sie eine einzigartige und momentan einzig realisierbare Lösung für die lange Betriebsdauer kleiner Satelliten im LEO darstellen. Das liegt an ihrem hohen spezifischem Impuls, der nicht durch die Verwendung chemischer Antriebe erreicht werden kann. Der spezifische Impuls $I_{sp}$ beschreibt wie viel Impulsänderung (Kraft) $F$ pro ausgestoßene Masse $\dot{m}$ erreicht werden kann und ist damit von der Austrittsgeschwindigkeit $v_e$ des Treibstoffes bestimmt und wird häufig auf die Erdbeschleunigung $g$ normiert :
\begin{equation}
    I_{sp} = \frac{F}{\dot{m}g} = \frac{v_e}{g}.
\end{equation} 
Der hohe spezifische Impuls elektrischer Triebwerke ist im Wesentlichen darauf zurückzuführen, dass die für die Beschleunigung eingesetzte Energie nicht aus dem Treibstoff selbst kommt, wie es bei chemischen Triebwerken der Fall ist, sondern über ein magnetisches Feld von außen eingespeist wird. So können viel höhere Austrittsgeschwindigkeiten erreicht werden. Damit sind elektrische Antriebe effizienter und können mit weniger Treibstoff mehr Impulsänderung erreichen, indem sie elektrische Energie aus, zum Beispiel, Solarzellen verwenden. Ein Blick auf die von William Moore und Konstantin Ziolkowski formulierte Raktengleichung veranschaulicht die Relevanz der Austrittsgeschwindigkeit. Die Gleichung beschreibt die Änderung der Geschwindigkeit eines Raumschiffs durch den Ausstoß von Masse. Sie lautet:
\begin{equation}
    \Delta v = v_e \cdot \ln\left(\frac{m_0}{m_f}\right),
\end{equation}
wobei $\Delta v$ die gesamte erreichbare Änderung der Geschwindigkeit beschreibt, $v_e$ die Austrittsgeschwindigkeit und $\frac{m_0}{m_f}$ das Verhältnis aus der Raketenmasse mit und ohne Treibstoff. Wie zu sehen ist, geht die Austrittsgeschwindigkeit propotional ein, während das Verhältnis der Massen logarithmisch eingeht. 

Ein solches elektrisches Antriebssysteme ist das Radiofrequenzionentriebwerk (RIT). Es wurde von Prof. Löb an der Justus Liebig Universität (JLU) Gießen in den 60er Jahren entwickelt und steht auch heute noch im Fokus der Forschung an der JLU. Das RIT ist neben dem Hall-Effekt-Thruster (HET) eines der effizientesten elektrischen Triebwerke und wurde bereits auf vielen erfolgreichen Missionen eingesetzt \cite[S. 6]{ion}. Die BepiColombo-Mission 2018 ist ein bekanntes Beispiel.

Das RIT ist ein sogenanntes Gitterionentriebwerk, welches auf der Ionisierung von Gasen basiert, welche dann elektrostatisch auf sehr hohe Austrittsgeschwindigkeiten beschleunigt werden. Dafür wird im Triebwerk ein Plasma gezündet, in dem über magnetische Wecheslfelder Energie auf freie Elektronen übertragen wird, die dann, Stoßionisationen durchführen können. Der historisch am häufigsten verwendete Treibstoff für das RIT ist Xenon. Xenon hat eine hohe Masse und ist ein reaktions-träges Edelgas. Es hat sich bereits bei einer Vielzahl von Missionen bewährt. Nachteilhaft ist allerdings, das die Gewinnung von Xenon äußert aufwendig ist. Sie basiert ausschließlich auf der Luftrennung und anschließenden Extraktion aus flüssigem Sauerstoff. Das macht Xenon zu einem sehr teuren Treibstoff, dessen Produktion nicht dem steigenden Bedarf nachkommen kann. Auch zu beachten, ist sein enormer Energiebedarf und damit verbundener hoher CO$_2$-Faktor von etwa 500 \cite{CO2}. Deswegen hat in den letzten Jahren die Forschung an alternativen Treibstoffen für elektrische Antriebe stark an Relevanz gewonnen. Alternative Treibstoffe sind vorallem für den Einsatz auf kleinen, kommerziellen Satelliten interessant. 

Herauszufinden, ob ein Treibstoff geeignet ist, ist allerdings nicht trivial. Es müssen viele Faktoren berücksichtigt werden, die sowohl mit den Eigenschaften des Treibstoffs selbst, mit den Prozessen bei der Ionisation als auch der Interaktion des Treibstoffs und Triebwerks zusammenhängen. Bevor extensive Tests mit dem Triebwerk direkt durchgeführt werden, kann die Untersuchung der Ionisationseigenschaften und Querschnitte bereits viele wichtige Informationen gewinnen und so Kandidaten früh ausschließen und Vorhersagen über die Effizienz treffen. Sehr wichtige Größen sind dabei die Ionisationsenergie, der Ionisationswirkungsquerschnitt, die Ionisationsrate und die Häufigkeitsverteilung der verschiedenen Ionen, die bei der Ionisation entstehen \cite{ion}. Diese Größen sind besonders für die Beschreibung des Plasmas und somit auch für die Optimierung des Antriebs entscheidend. Ein idealer Treibstoff sollte eine niedrige Ionisationsschwelle und einen hohen Ionisationsgrad haben, da so minimale Energie benötigt wird, um ein dichtes Plasma zu erzeugen \cite{Prop}. Die Häufigkeitsverteilung der Ionen sollte sich auf möglichst wenige Spezies konzentrieren, da die Extraktion der Ionen für ein Masse-zu-Ladungsverhältnis optimiert wird.

Eine Untersuchung der Ionisationseigenschaften eines Gases erweist sich als komplex. Damit Ionisationsprozesse genau untersucht werden können, muss die Ionisation in einem sehr kontrollierten Umfeld stattfinden. Das bedeutet vorallem, dass die Interaktionspartner und ihre Eigenschaften genau bekannt sein müssen. Die dabei erzeugten Ionen müssen einzeln nachgewiesen werden, damit ihre Häufigkeit akkurat bestimmt werden kann. 

\section{Zielsetzung}
Basierend auf früheren Arbeiten der Arbeitsgruppe für Ionentriebwerke an der JLU, wird in dieser Arbeit die bestehen Anlage \textit{ZeroB} ausgebaut, um die soeben genannten relevanten Größen verschiedener Gase zu bestimmen. Der Einbau eines neuen Detektors soll die Aussagekraft der Messungen verbessern. Mithilfe einer ionenoptischen Simulation sollen diese überprüft werden und mögliche Optimierungen herausgearbeitet werden. Mit der Simulation soll die theoretisch mögliche Auflösung der Anlage ermittelt werden. Ziel ist es die Ergebnisse der Simulation mit den Messwerten zu vergleichen. 

Im Anschluss wird ein Überblick über verwandte Arbeiten gegeben, um diese Arbeit in den wisschenschaftlichen Kontext einzugliedern. 